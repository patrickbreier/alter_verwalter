%%
% German Latex Letter Template 
% Use if you want to crate a Letter in DIN A4. You can use it in English and German 
% as well, just set language at begining of plain text.
% Created by Jan Boelmann @ Nov. 2016 Jan.boelmann@live.de
%%

\documentclass[
    sender,
    paper=a4,
    version=last,
    fontsize=11pt,
    DIV=13,
    BCOR=0mm]{scrlttr2}
\parskip3mm
\parindent0mm % if you want to have no lineskip
\usepackage[english,ngerman]{babel}
\usepackage[utf8]{inputenc}
\usepackage[T1]{fontenc}
\usepackage{csquotes}
\usepackage{enumitem}  % for customized enumerate
\usepackage{courier}
% Set Font: sans serif Latin Modern
\usepackage{lmodern}
% Set Page layout:

\usepackage{changepage}
%\changepage{text height}{text width}{even-side margin}
%{odd-side margin}{column sep.}
%{topmargin}{headheight}{headsep}{footskip}
\changepage{+3cm}{}{}{}{}{}{}{}{-5cm}
\LoadLetterOption{sender}
\renewcommand\familydefault{\ttdefault}


\begin{document}
% Set Appendix text at very end (colon will be set automatically)
\setkomavar*{enclseparator}{Appendix}
% subject, date, place:
\setkomavar{subject}{Betreff: Alter Verwalter - Wir werden 30!}
\setkomavar{date}{\today}
\setkomavar{place}{Mond}


% Set recipient of letter
\begin{letter}{
    <name>\\
    <adress>\\
    }
    
\opening{Sehr geehrtes Mitglied unserer Feiergemeinschaft,}
% Write here your Letter text. You can choose here the language for typeset. ("english", or "ngerman")
\selectlanguage{ngerman}
%\parindent2mm

die befreundeten Personen Baum, Haus und Busch - im Nachfolgenden Gastgebende genannt 
- erreichen das Alter von 30 Jahren\footnote{Baum: +31 Tage; Haus +16 Tage; Busch +2 Tage} und laden Sie hiermit herzlich zu ihrer Geburtstagsfeier ein.

Diese findet statt:

am 28. Januar 2023 \\
in der Straße 1, 00000 Mond. \\
Einlassbeginn 19 Uhr.

Es wäre den Gastgebenden eine außerordentliche Freude, wenn du persönlich an der Zusammenkunft teilnehmen könntest. Im Falle einer Nichtteilnahme wird um schriftliche Rückmeldung in Form des beigefügten Antrags auf Befreiung von der Teilnahmepflicht gebeten um unentschuldigtem Fernbleiben vorzubeugen.
Eine Abschrift dieser Einladung kann bei Verlust gegen Vorschuss eines selbsthaftenden Wertzeichens\footnote{Briefmarke} durch das zuständige \textit{Verwalterungsamt} erneut ausgestellt werden.
Um den geregelten Ablauf der Veranstaltung zu gewährleisten, ist es imperativ, dass du dich zu der auf der Eintrittskarte indizierten Zeit am o.g. Ort einfindest. Sofern nicht anders vereinbart, wird um unaufgefordertes Vorzeigen eines Identitätsnachweises gebeten.

%\selectlanguage{german}%Choose language for closing text
\closing{Mit freundlichen Grüßen} % use "Mit freundlich Grüßen" i.e.
\begin{center}
    - Dieses Schreiben wurde maschinell erstellt und ist ohne Unterschrift gültig -
\end{center}
%\encl{\begin{itemize}[noitemsep,topsep=0pt,parsep=0pt,partopsep=0pt]
%\item [--] Personalisierte Eintrittskarte
%\item [--] Vorduck für Antrag auf Befreiung von der Teilnahmepflicht
%\item [--] Grußkarte
%\end{itemize}}
\vspace*{\fill}

% end of letter
\end{letter}

\end{document}